\documentclass[a4paper,11pt]{report}
\usepackage[italian]{babel}
\usepackage[utf8]{inputenc}
\usepackage[total={170mm,267mm},top=15mm,bottom=15mm,left=21mm,right=21mm]{geometry}

\begin{document}

\begin{titlepage}
  \clearpage\thispagestyle{empty}
  \centering
  \vspace{1cm}
  {\normalsize Informatica - Area scientifica \\  Dipartimento di Scienze matematiche, informatiche e multimediali\\  Università di Udine \par}
  \vspace{3cm}
  {\Huge \textbf{Progetto di Social Computing}}\\
  \vspace{4cm}
  {\Large Arzon Francesco ()\\ Galvan Matteo (142985)\\ Parata Loris (144338)\\ Lorenzo -------- ()}\\
  \vspace{12cm}
  {\normalsize Anno accademico 2020/2021}
  \pagebreak
\end{titlepage}

\tableofcontents{}
\pagebreak

\chapter{Introduzione}
Il primo progetto di Social Computing consiste nello studio della rete sociale di 5 utenti di Twitter.\\
Lo studio è stato svolto mediante la costruzione di un grafo, rappresentante  rete sociale, costituto dai 5 utenti principali e della loro relativa rete di contatti costituita dai loro follower, following e da rispettivi sottoinsiemi campionati in maniera random.
In specifico abbiamo analizzato la relazione diretta di \textbf{follow} tra tutti i nodi del grafo ed i 5 profili scelti.

\pagebreak

\chapter{Costruzione del grafo}
\section{Download dei nodi}
Il primo passo consiste nello scaricare tutti i followers attraverso la api.followers di Twitter dei cinque nodi principali:

\begin{itemize}
\item @Mizzaro
\item @damiano10
\item @Miccighel\_
\item @eglu81
\item @KevinRoitero
 \end{itemize}
json e compagnia bella con codice se serve \\
\setlength{\parindent}{0pt} 
Successivamente abbiamo scelto 5 followers e 5 following randomicamente per ognuno dei 5 account. In seguito, da ognuno di essi sono stati scelti altri 10 account followers e 10 account following sempre in maniera casuale.\\

Infine, una volta ottenuti tutti gli account, abbiamo scaricato tutte le informazioni principali relative agli account mediante l'api.get user.\\
Per un totale di 3101 nodi.
\section{Creazione degli archi}
Successivamente abbiamo controllato l'esistenza di una relazione tra tutti gli account scaricati ed i 5 nodi principali con la funzione api.show friendship. Aggiungendo gli archi raffiguranti l'azione di follow al grafo.
\subsection{Ottimizzazione archi}
E' possibile rilevare tutti i nodi direttamente connessi ai 5 account andando a visualizzare direttamente i rispettivi followers, riducendo significativamente i costi in termine di richieste all'API. Ma per attinenza alla traccia abbiamo fatto un controllo completo per ogni nodo scaricato precedentemente.

\section{Visualizzazione del grafo}
La visualizzazione interattiva del grafo costruito con le funzioni messe a disposizione di networkX avviene utilizzando la libreria apposita pyvis.
\subsection{Ottimizzazione visualizzazione}
E' possibile ridurre i costi per l'elaborazione grafica di costruzione del grafo impostando il parametro opzionale phisic = False. Questo parametro a discapito dell'interazione fisica nel trascinamento  dei nodi che avrebbero una risposta fisica, permette di risparmiare l'80 percento di tempo.

\pagebreak
\chapter{Analisi del grafo}
Applicando le relative funzioni messe a disposizione dalla libreria di networkX abbiamo potuto stabilire che il grafo è:
\begin{itemize}
	\item Il grafo da noi analizzato non è connesso.\\
	E' errato pensare che tutti gli utenti che seguono l'account \textbf{UtenteTwitter} a loro volta hanno dei followers che seguono \textbf{UtenteTwitter}. Nel caso in cui tenessimo traccia delle relazioni interne tra i nodi di secondo livello e quelli di terzo livello, visualizzando i path indiretti, allora sarebbe risultato connesso. Ma questo dipende dalla componente casuale che sceglie i nodi da cui scaricare i relativi follower dei follower. 
	\item Il grafo è ..
	\item Il grafo è ..
	\item Il grafo è ..
	\item Il grafo è ..
	\item Il grafo è ..
\end{itemize}
Analizzando il sottografo dell'account KevinRoitero:

\pagebreak

\chapter{CONCLUSIONE}
\pagebreak

\end{document}\\