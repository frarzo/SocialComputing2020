\documentclass[a4paper,11pt]{report}
\usepackage[italian]{babel}
\usepackage[utf8]{inputenc}
\usepackage[total={170mm,267mm},top=15mm,bottom=15mm,left=21mm,right=21mm]{geometry}

\begin{document}

\begin{titlepage}
  \clearpage\thispagestyle{empty}
  \centering
  \vspace{1cm}
  {\normalsize Informatica - Area scientifica \\  Dipartimento di Scienze matematiche, informatiche e multimediali\\  Università di Udine \par}
  \vspace{3cm}
  {\Huge \textbf{Progetto di Social Computing}}\\
  \vspace{4cm}
  {\Large Arzon Francesco ()\\ Galvan Matteo (142985)\\ Parata Loris (144338)\\ Lorenzo -------- ()}\\
  \vspace{12cm}
  {\normalsize Anno accademico 2020/2021}
  \pagebreak
\end{titlepage}

\tableofcontents{}
\pagebreak

\chapter{Introduzione}
Il primo progetto di Social Computing consiste nello studio della rete sociale di alcuni utenti di Twitter.\\
Lo studio è stato eseguito mediante la costruzione di un grafo rappresentante la rete sociale, costituto dagli utenti principali e dallo studio delle relazioni che esistono tra essi ed i loro rispettivi follower.

\pagebreak

\chapter{Costruzione del grafo}
\section{Download dei nodi}
Il primo passo consiste nello scaricare tutti i followers attraverso la api.followers di Twitter dei cinque nodi principali:

\begin{itemize}
\item @Mizzaro
\item @damiano10
\item @Miccighel\_
\item @eglu81
\item @KevinRoitero
 \end{itemize}
json e compagnia bella con codice se serve \\
\setlength{\parindent}{0pt} 
Successivamente abbiamo scelto 5 followers randomicamente da ognuno dei 5 account, e da ognuno di essi sono stati scelti altri 10 account sempre in maniera casuale.\\

codice questo     \^ su \\
Infine, una volta ottenuti tutti gli account, abbiamo scaricato tutte le informazioni principali relative agli account mediante l'api.get user.\\
Per un totale di  .... nodi.
\section{Creazione degli archi}
Successivamente abbiamo controllato l'esistenza di una relazione tra tutti gli account scaricati ed i 5 nodi principali con la funzione api.show friendship. Aggiungendo gli archi raffiguranti l'azione di follow al grafo.

\section{Visualizzazione del grafo}
La visualizzazione interattiva del grafo costruito con le funzioni messe a disposizione di networkX avviene utilizzando la libreria apposita pyvis.
\pagebreak
\chapter{Analisi del grafo}
Applicando le relative funzioni messe a disposizione dalla libreria di networkX abbiamo potuto stabilire che il grafo è:
\begin{itemize}
	\item Il grafo è ..
	\item Il grafo è ..
	\item Il grafo è ..
	\item Il grafo è ..
	\item Il grafo è ..
	\item Il grafo è ..
\end{itemize}
Analizzando il sottografo dell'account damiano10 :

\pagebreak

\chapter{CONCLUSIONE}
\pagebreak

\end{document}\\