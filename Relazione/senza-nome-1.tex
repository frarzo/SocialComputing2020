\documentclass[a4paper,11pt]{report}
\usepackage[italian]{babel}
\usepackage[utf8]{inputenc}
\usepackage[total={170mm,267mm},top=15mm,bottom=15mm,left=21mm,right=21mm]{geometry}

\begin{document}

\begin{titlepage}
  \clearpage\thispagestyle{empty}
  \centering
  \vspace{1cm}
  {\normalsize Informatica - Area scientifica \\  Dipartimento di Scienze matematiche, informatiche e multimediali\\  Università di Udine \par}
  \vspace{3cm}
  {\Huge \textbf{Progetto di Social Computing}}\\
  \vspace{4cm}
  {\Large Arzon Francesco ()\\ Galvan Matteo (142985)\\ Parata Loris (144338)\\ Lorenzo -------- ()}\\
  \vspace{12cm}
  {\normalsize Anno accademico 2020/2021}
  \pagebreak
\end{titlepage}

\tableofcontents{}
\pagebreak

\chapter{INTRODUZIONE}
Il primo progetto di Social Computing consiste nello studio della rete sociale di alcuni utenti di Twitter mediante l'utilizzo di un grafo, e dello studio delle relazioni che esistono tra gli utenti.

\pagebreak

\chapter{TITOLO BELLO che madonna altro che 5 punti, ne sgancia 30}
Il primo passo è stato scaricare attraverso api.followers e api.friends le informazioni dei 5 account:

\begin{itemize}
\item @Mizzaro
\item @damiano10
\item @Miccighel\_
\item @eglu81
\item @KevinRoitero
 \end{itemize}
json e compagnia bella con codice se serve \\
\setlength{\parindent}{0pt} 
Vengono poi scelti 5 followers randomicamente da ognuno dei 5 account: da questi account se ne scaricano 10. \\
Opero la stessa azione anche per i following: vengono scelti 5 followers randomicamente da ognuno dei 5 account: da questi account se ne scaricano 10.

codice questo     \^ su


\pagebreak

\chapter{CONCLUSIONE}
\pagebreak

\end{document}